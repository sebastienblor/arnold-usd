
\section{Creating a Light Filter Shader}

To create this shader you will need, as previously, a source code file, a loader, a metadata and a template.
We will take a look at them here.

\subsection{Creating the shader}

We can add the light filter shader to the previous shader loader we created in the previous sections:

First, create the shader:

\inputminted[mathescape,
linenos,
numbersep=5pt,
frame=lines,
framesep=2mm,
baselinestretch=1,
fontsize=\footnotesize,
tabsize=3,
label=simpleLightFilter.cpp]
{c++}{simpleLightFilter.cpp}

Now, to add the shader to the loader, you only need to add this code to the previous \texttt{Loader.cpp} file from the previous sections:

\begin{minted}[mathescape,
numbersep=5pt,
frame=lines,
framesep=2mm,
baselinestretch=1,
fontsize=\footnotesize,
label=loader.cpp,
tabsize=3]
{c++}
...
extern AtNodeMethods* SimpleLightFilterMtd;

enum{
   MY_SHADER_1 = 0,
   MY_SHADER_2,
   PARAMETERS_SHADER,
   SIMPLE_LIGHT_FILTER
};
...

   case SIMPLE_LIGHT_FILTER:
      node->methods      = SimpleLightFilterMtd;
      node->output_type  = AI_TYPE_RGB;
      node->name         = "simpleLightFilter";
      node->node_type    = AI_NODE_SHADER;
      break;
...

\end{minted}

You will be able to compile the light filter and the loader again as described in the previous sections,
check that Arnold can load the shader correctly, and copy the compiled shader to the
correct folder (as if it was a normal shader) so that Maya can use it.

\subsection{Integrating the light filter in Maya}

To integrate a light filter in Maya, we need to add extra information to the
metadata file and create a template script as in the previous section.

\subsubsection{Adding light filter metadata information}

To add the metadata information, you only have to add this to the \texttt{loader.mtd} file created in the
previous sections that you placed in the same folder as the compiled shader file.

\inputminted[mathescape,
linenos,
numbersep=5pt,
frame=lines,
framesep=2mm,
baselinestretch=1,
fontsize=\footnotesize,
tabsize=3,
label=loader.mtd,
firstline=101,
firstnumber=101]
{mtd}{loader3.mtd}

Here it is important to set \texttt{maya.classification} to \texttt{"light/filter"}. And also
set \texttt{maya.lights} to a string that contains all the lights that can use this light filter
like: \texttt{ambientLight}, \texttt{directionalLight}, \texttt{pointLight},
\texttt{spotLight}, \texttt{areaLight} or \texttt{aiAreaLight}.

\subsubsection{Adding a Maya template}

As in the previous sections, we can create a simple template for this light filter in the same was as for
normal shaders. Anyway, we will avoid some of the commands that do not make sense in a light filter:

\inputminted[mathescape,
linenos,
numbersep=5pt,
frame=lines,
framesep=2mm,
baselinestretch=1,
fontsize=\footnotesize,
tabsize=3,
label=aiSimpleLightFilterTemplate.py]
{python}{aiSimpleLightFilterTemplate.py}

Now, when you create a type of light that is accepted by this light filter, you will be able to
add this light filter in the "Light Filters" section in the "Arnold" section of the light:

\begin{figure}[H]
\centering
\colorbox{black}{\includegraphics[width=9cm]{LightFiltersMenu.png}}
\caption{Light Filters Menu}
\label{LightFiltersMenu}
\end{figure}





