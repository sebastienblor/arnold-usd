\section{Creating a Shader Loader}


If you write several shaders, it could be useful to create a loader that will load all of them from the same library file.
It is also possible to create a metadata file with all the metadata information of all the shaders.

\subsection{Two basic shaders}

The first thing you need are the shaders you want to include; here are two very basic ones:

\inputminted[mathescape,
linenos,
numbersep=5pt,
frame=lines,
framesep=2mm,
baselinestretch=1,
fontsize=\footnotesize,
tabsize=3,
label=myShader1.cpp]
{c++}{myShader1.cpp}

$ $\\

\inputminted[mathescape,
linenos,
numbersep=5pt,
frame=lines,
framesep=2mm,
baselinestretch=1,
fontsize=\footnotesize,
tabsize=3,
label=myShader2.cpp]
{c++}{myShader2.cpp}

\subsection{The loader}

With these two shaders, it will be easy to write a loader for them:

\inputminted[mathescape,
linenos,
numbersep=5pt,
frame=lines,
framesep=2mm,
baselinestretch=1,
fontsize=\footnotesize,
tabsize=3,
label=loader.cpp]
{c++}{loader.cpp}

When you have these files, you can compile them in a similar way to a single shader:

\begin{description}
\item[In Windows] \hfill \\
Open a Visual Studio command prompt and execute the following:\\
\verb|cl /LD /I %ARNOLD_PATH%\include /EHsc|\\
\verb|       loader.cpp myShader1.cpp myShader2.cpp|\\
\verb|       /link /LIBPATH:%ARNOLD_PATH%\lib ai.lib|\\
Or you can compile each shader separately and then link them to the loader:\\
\verb|cl /c /I %ARNOLD_PATH%\include /EHsc myShader1.cpp|\\
\verb|cl /c /I %ARNOLD_PATH%\include /EHsc myShader2.cpp|\\
\verb|cl /LD /I %ARNOLD_PATH%\include /EHsc loader.cpp|\\
\verb|       myShader1.obj myShader2.obj|\\
\verb|       /link /LIBPATH:%ARNOLD_PATH%\lib ai.lib|\\
\end{description}


Now you can, as with your first shader, use \texttt{kick} with the \texttt{-info} option to check if they have been correctly created and loaded by Arnold.

{\footnotesize \begin{verbatim}
>> %ARNOLD_PATH%\bin\kick -l <path_to_shader> -info myShader1

node:         myShader1
type:         shader
output:       RGB
parameters:   2
filename:     <path_to_shader>/loader.dll
version:      X.X.X.X

Type          Name                              Default
------------  --------------------------------  --------------------------
RGB           color                             0, 0, 0
STRING        name                              default_name

>> %ARNOLD_PATH%\bin\kick -l <path_to_shader> -info myShader2

node:         myShader2
type:         shader
output:       RGB
parameters:   2
filename:     <path_to_shader>/loader.dll
version:      X.X.X.X

Type          Name                              Default
------------  --------------------------------  --------------------------
RGB           color                             1, 1, 1
STRING        name                              default_name
\end{verbatim}}


Finally, to make these shaders availables to MtoA, you should copy the compiled file to this location:\\
\verb|%MTOA_PATH%\shaders\|\\

\subsection{Add the metadata file}

To make MtoA recognize metadata information for the created shaders, you can create a metadata file for all the shaders.

A simple metadata file for these shaders could be like this:

\inputminted[mathescape,
linenos,
numbersep=5pt,
frame=lines,
framesep=2mm,
baselinestretch=1,
fontsize=\footnotesize,
tabsize=3,
label=loader.mtd]
{mtd}{loader.mtd}

Then copy this file to the same folder where the compiled shaders are:

\texttt{\%MTOA\_PATH\%\textbackslash shaders\textbackslash}\\

Remember that the metafile name should be the same as the compiled shader; if for example, your loader is called \texttt{loader.dll}, the metadata file must be \texttt{loader.mtd}.
